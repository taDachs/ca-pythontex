\documentclass[11pt,a4paper]{article}

\usepackage[T1]{fontenc}
\usepackage[ngerman]{babel}
\usepackage[utf8]{inputenc}

\usepackage{pythontex}
\usepackage{xcolor}

\usepackage{amsmath}
\usepackage{float}
% \usepackage{mathtools}

\usepackage[tt=false]{libertine}   % !!!!! das muss man nicht nutzen
\usepackage[libertine]{newtxmath}  % !!!!! das muss man nicht nutzen
%\usepackage[supstfm=libertinesups,supscaled=1.2,raised=-.13em]{superiors} % params taken from doc

%
%\usepackage{tgpagella}
%\usepackage[euler-digits]{eulervm}
%
\usepackage{csquotes}

\usepackage{microtype}

\usepackage{fancyvrb}

%\usepackage{graphicx}  % !!!!! benutzen, wenn Grafiken eingebunden werden sollen

\usepackage{booktabs}
\usepackage[shortlabels]{enumitem}
\setlist{noitemsep}

\usepackage{titlesec}
\usepackage{tcolorbox}
\tcbuselibrary{listingsutf8}

\usepackage{bbold}
\newcommand{\Z}{\mathbb{Z}}

\usepackage{hologo}

\definecolor{codegreen}{RGB}{0, 128, 0}
\definecolor{codegray}{rgb}{0.5,0.5,0.5}
\definecolor{codepurple}{rgb}{0.58,0,0.82}
\definecolor{codered}{RGB}{192, 53, 53}

\lstdefinestyle{mystyle}{
    commentstyle=\color{codegreen},
    keywordstyle=\color{codegreen},
    numberstyle=\tiny\color{codegray},
    stringstyle=\color{codered},
    basicstyle=\ttfamily,
    breakatwhitespace=false,
    breaklines=true,
}

\lstset{style=mystyle}

%-----------------------------------------------------------------------------
% für das Deckblatt

\usepackage{tikz}

\newcommand{\teilnehmername}{Maximilian Schik} % !!!!!
\newcommand{\teilnehmermatrnr}{2209036}        % !!!!!
\newcommand{\seminarart}{\LaTeX-Projektarbeit} % !!!!!
\newcommand{\seminarlp}{2 LP}                  % !!!!!
\newcommand{\seminarjahr}{2022}                % !!!!!
%-----------------------------------------------------------------------------
\newcommand{\meta}[1]{$\langle$\textit{#1}$\rangle$}
\newcommand{\paket}[1]{\texttt{#1}}
\newcommand{\prgname}[1]{\texttt{#1}}

%-----------------------------------------------------------------------------
\author{Maximilian Schik}
\title{PythonTex}

%-----------------------------------------------------------------------------
\pyc{from typing import List}
\newcommand{\tableCA}[3]{\py{rule(#1, #2, "#3")}}
\newcommand{\tikzCA}[4]{\py{rule_tikz(#1, #2, "#3", #4)}}
\newcommand{\toUpperCase}[1]{\py{"#1".upper()}}

\newcommand{\pythontex}{\textit{Python\TeX{}}}
\newcommand{\todo}[1]{\marginpar{\textcolor{red}{\textbf{TODO:} #1}}}

%=============================================================================
\begin{document}
%=======================================================================
% Anfang erste Seite
{\thispagestyle{empty}\large\sffamily\raggedright
%
\begin{tikzpicture}[remember picture,overlay]
  \coordinate[xshift=5mm,yshift=-5mm] (NW) at (current page.north west) {};
  \coordinate[xshift=-5mm,yshift=-5mm] (NE) at (current page.north east) {};
  \coordinate[xshift=-5mm,yshift=13mm] (SE) at (current page.south east) {};
  \coordinate[xshift=5mm,yshift=13mm] (SW) at (current page.south west) {};

  \draw[line width=0.25pt] (NW)
    [rounded corners=5mm] -- (NE)
    [sharp corners] -- (SE)
    [rounded corners=5mm] -- (SW)
    [sharp corners] -- cycle
  ;
\end{tikzpicture}
%
\unskip % keine Ahnung warum das nötig ist
\noindent \textbf{\Large \seminarart\ (\seminarlp)}
\\[\baselineskip]
%
% \\[1ex]
%
im Sommersemester \seminarjahr

\vspace*{3\baselineskip}

\noindent \textbf{\Large Pyt"|hon\TeX{}} \\[\baselineskip]
%
von \textbf{\teilnehmername}, Matr.nr.~\teilnehmermatrnr

\vspace*{3\baselineskip}

% \noindent \textbf{\Large PythonTex} \\[\baselineskip]
%
% nachfolgend ein Beispiel für Pro-/Seminare, für Konferenzbeiträge, Buchausschnitte, ...
% für LaTeX bitte was passendes ersetzen !!!!!
%
Thomas Worsch\\[1ex]
%
\LaTeX, \texttt{beamer}, \texttt{tikz} und Co.\\[1ex]
%
% Zentralblatt für historischen ÖPNV, Band \textbf{42}, S.~123-456
}
\clearpage
% Ende erste Seite
%=======================================================================
% Anfang zweite Seite
{\thispagestyle{empty}\raggedright

\noindent \textbf{\Large Erklärung}\\[1ex]
gemäß \S 6 (7) der Prüfungsordnung Informatik (Bachelor) 2015: % !!!!! oder (Master)    oder \S 6 (11) (Bachelor) 2008
\\[\baselineskip]

\noindent
Ich versichere wahrheitsgemäß, die vorliegende Arbeit selbstständig
angefertigt, alle benutzten Hilfsmittel vollständig und genau
angegeben und alles kenntlich gemacht zu haben, was aus Arbeiten
anderer unverändert oder mit Abänderungen entnommen wurde.

\vspace*{30mm}
\noindent
\begin{tabular}{@{}l}
  \hline
   \\[-1ex]
  \hbox to 0.6\textwidth{(\teilnehmername, Matr.nr.~\teilnehmermatrnr) \hss}
\end{tabular}
}
\clearpage
% Ende zweite Seite
%=======================================================================

%-----------------------------------------------------------------------------
\section{Einleitung}
\pythontex ist ein \LaTeX{} Paket das es einem erlaubt Python Code innerhalb eines \LaTeX{} Dokumentes zu verwenden und dessen Ausgabe auch in dem Dokument darzustellen.
%
Das Paket erlaubt es dem Nutzer den Code direkt neben seiner Ausgabe zu speichern.
%
Dadurch kann man Dokumente verbreiten die direkt den Code beinhalten, mit dem die Grafiken oder sonstige Daten im Dokument selbst erzeugt wurden.
%
Das erzeugt einen hohen Grad an Reproduzierbarkeit.
%
Normalerweise wird Reproduzierbarkeit nur erreicht, indem man den verwendeten Programm Code seperat verbreitet.
%
\\
%
Die Einbindung des Codes in das Dokument ist vielfältig, so kann man zum Beispiel mit Hilfe des \verb+\py+ Befehls direkt Python Code im Dokument ausführen.
%
\verb#\py{1 + 2 + 3}# gibt einem als Ausgabe \texttt{\py{1 + 2 + 3}}. Es erlaubt auch direkt \LaTeX{} Befehle mit auszugeben.
%
\verb#\py{r"\textit{Hello World}"}# ergibt \py{r"\textit{Hello World}"} (beachte die \textit{Italics}).
%
\\
%
\pythontex{} wurde hauptsächlich von Geoffrey M Poore entwickelt und maintained.
%
Es ist verfügbar in \texttt{\TeX{} Live} und \texttt{MiK\TeX{}}.
%
\todo{quelle für studie über Reproduzierbarkeit, sowas wie https://elifesciences.org/articles/23383 aber mehr computer, weniger krebs}

\section{Das Paket \pythontex{}}
\pythontex{} stellt eine Reihe an nützlichen Umgebungen und Befehlen bereit, hier werden allerdings nur die wichtigsten behandelt.
%
Die wichtigsten sind:
%
\begin{itemize}
  \item \verb#\py# - Führt den gegebenen Code aus und gibt die Ausgabe zurück.
  \item \verb#\pyc# - Führt den gegebenen Code aus, gibt nur \pyb{print} Aufrufe aus. Im Gegensatz zu \verb#\py# wird nicht die direkte Ausgabe mit zurückgegeben.
  \begin{itemize}
    \item \verb#\py{"foo".upper()}# -> \py{"foo".upper()}
    \item \verb#\pyc{"foo".upper()}# -> Keine Ausgabe, um die gleiche Ausgabe zu erhalten müsste \verb#\pyc{print("foo".upper())}# aufgerufen werden.
  \end{itemize}
    \todo{fix arrows}
  \item \verb#\pyb# - Führt den gegebenen Code aus und formatiert ihn. Die Ausgabe kann später mittels \verb#\printpythontex# ausgegeben werden.  \todo{pyb klingt kacke, zweimal ausgabe}
\end{itemize}
%
Es gibt noch einige Umgebungen, diese entsprechen teilweise einem der vorangegangen Befehle.
\begin{itemize}
  \item \texttt{pycode} - führt den gegebenen Code aus. Gibt nur \pyb{print} Aufrufe aus.
  \item \texttt{pyblock} - führt den gegebenen Code aus und stellt den Code als Listing mittels \texttt{pygments} dar. Ähnlich zu \verb#\pyb# kann mit \verb#\printpythontex# die Ausgabe ausgegeben werden.
  \item \texttt{pyconsole} - führt den gegebenen Code in einer interaktiven Konsolen Sitzung aus.
\end{itemize}
%
Man kann auch Python Aufrufe in einen \LaTeX{} Befehl kapseln.
%
So würde zum Beispiel \verb~\newcommand{\toUpperCase}[1]{\py{"#1".upper()}}~ einen Befehl erzeugen, der eine beliebige Zeichenkette nimmt und diese in eine Zeichenkette aus Großbuchstaben umformt.
%
Diesen Befehl kann man dann einfach mit \verb#\toUpperCase{foo}# aufrufen um die Ausgabe ``\toUpperCase{foo}`` zu erhalten.
%
Dabei sollte man stehts beachten das sich sämtlicher Python Code einen Namespace teilt, daher sollte man nach Möglichkeit Logik immer in Funktionen kapseln.
%
Ein weiterer Nachteil ist von \pythontex{} ist ein Python Problem.
%
Das folgende Code Stück würde zu einem Fehler führen wenn man es als normales Python Programm ausführt.
%
\begin{lstlisting}[language=Python]
foo()

def foo():
  print("bar")
\end{lstlisting}
%
Das Problem ist das Python kein Hoisting hat (Deklarationen werden an die Spitze ihres Scopes gezogen).
%
Daher ist es manchmal etwas schwer die Ausgabe eines Listings vor dem Listing selbst zu haben (als Beispiel mal Abbildung \ref{fig:rule_30_table} und Listing \ref{lst:rule_table}).
%
Man kann zwar eine Kopie des Codes in das Präemble schreiben, allerdings endet das in dupliziertem Code, genau das was \pythontex{} verhindern will.
%
\\
%
Die grundlegene Benutzung von \pythontex{} besteht aus wenigen Schritten.
%
Man verwendet das \pythontex{} Paket mittels \verb#\usepackage{pythontex}# in einem Dokument und ruft Python mit einem der beschriebenen Befehle innerhalb des Dokuments auf.
%
Dieses kompiliert man dann mit einer der unterstützten \LaTeX{} Engines (pdf\TeX{}, Xe\TeX{} oder Lua\TeX{}).
%
Danach ruft man \texttt{pythontex} auf das Dokument auf.
%
Das nimmt dann den Python Code aus dem Dokument und führt ihn aus.
%
Die Ergebnisse werden in einem Unterverzeichnis gespeichert.
%
Mit einem erneuten Aufruf der \LaTeX{} Engine werden nun diese Ergebnisse in das Dokument eingebunden und die finale PDF wird erzeugt.
%
Für dieses Dokument wurde das alles in einer Makefile zusammengefasst, welche in Listing \ref{lst:makefile} zu sehen ist.
%
\begin{listing}
  \begin{lstlisting}[language=make,columns=flexible]
build:
	pdflatex -interaction=nonstopmode sem-aus.tex
	pythontex sem-aus.tex
	pdflatex -interaction=nonstopmode sem-aus.tex
  \end{lstlisting}
  \caption{Makefile für dieses Dokument}
  \label{lst:makefile}
\end{listing}
%
\section{Ein paar Beispiele}
\subsection{Zelluläre Automaten als Tabelle}
Ein weiterer Vorteil von \pythontex{} ist die Möglichkeit, das Ergebnis von Algorithmen/Simulationen direkt zu visualisieren.
%
Man kann zum Beispiel die ersten Generationen eines elementaren Zellulären Automatens in einer Tabelle darstellen.
%
Dies ist möglich da \pythontex{} die Ausgabe direkt als \LaTeX{} interpretiert und demnach auf formattiert.
%
So kann sich innerhalb einer \texttt{pyblock} Umgebung eine Funktion definieren die einen Zellulären Automaten simuliert und die Genrationen danach in eine Tabelle packt.
%
Ein mögliches Beispiel dafür wäre Listing \ref{lst:rule_table}, das Ergebnis ist in Abbildung \ref{fig:rule_30_table} zu sehen.
%
Für alle Python Listings wurde die \texttt{pyblock} Umgebung verwendet.
%
\begin{listing}
  \centering
  \begin{pyblock}
def rule(n: int, t: int, seed: str) -> str:
    seed = "0" * t + seed + "0" * t
    rule = f"{n:b}".zfill(8)[::-1]
    gens = [seed]

    out = f"\\begin{{tabular}}{{ {'c ' * len(seed)} }}\n"
    for i in range(t):
        prev_gen = gens[i]
        new_gen = []
        for j in range(1, len(prev_gen) - 1):
            new_gen.append(rule[int(prev_gen[j-1:j+2], 2)])

        new_gen = "".join(new_gen)
        gens.append(new_gen)

    for i, gen in enumerate(gens):
        out += " &  " * i
        out += " & ".join(gen)
        out += " &  " * i + r"\\" + "\n"

    out += r"\end{tabular}"
    return out
  \end{pyblock}
  \caption{Python Code zum Erzeugen einer Tabelle, die die ersten \texttt{t} Generationen des eines Zellulären Automatik für eine gegebene Regel \texttt{n} mit dem Startwert \texttt{seed} enthält.}
  \label{lst:rule_table}
\end{listing}
%
\begin{figure}[H]
  \centering
  \tableCA{30}{3}{00101101001}
  \caption{Ausgabe von Listing \ref{lst:rule_table} mit \texttt{n = 30}, \texttt{t = 3} und \texttt{seed = "00101101001"}. Der \LaTeX{} Befehl war \texttt{\textbackslash tableCA\{30\}\{3\}\{00101101001\}}.}
  \label{fig:rule_30_table}
\end{figure}

\subsection{Fortgeschrittene Beispiele}
Alternativ kann man auch anstatt einer Tabelle ein \texttt{tikz} Bild zeichnen.
%
Eine Beispiel dafür gibts es in Listing \ref{lst:rule_tikz} zu sehen.
%
Das \texttt{tikz} Bild kann man dann mit dem \LaTeX{} Befehl \texttt{\textbackslash tikzCA\{<Regel>\}\{<Num. Gens>\}\{<seed>\}\{<Skalierung>\}} aufrufen, welcher ähnlich zu \verb#\tableCA# gebunden wurde.
%
\begin{listing}
  \centering
  \begin{pyblock}
def rule_tikz(n: int, t: int, seed: str, scale: float = 1) -> str:
  rule = f"{n:b}".zfill(8)[::-1]
  gens = [seed]

  for i in range(t):
      prev_gen = "0" + gens[i] + "0"
      new_gen = []
      for j in range(1, len(prev_gen) - 1):
          new_gen.append(rule[int(prev_gen[j-1:j+2], 2)])

      new_gen = "".join(new_gen)
      gens.append(new_gen)

  out = "\\begin{tikzpicture}\n"
  out += f"\\draw[step={scale}] (0,0)" \
         + f"grid ({len(seed) * scale},{(t + 1) * scale});\n"
  for i, gen in enumerate(gens):
      for j, cell in enumerate(gen):
          if int(cell):
              y = (t - i) * scale
              x = j * scale
              out += f"\\fill[black] ({x}, {y})" \
                     + f" rectangle ({x + scale},{y + scale});\n"

  out += "\\end{tikzpicture}\n"
  return out
  \end{pyblock}
  \caption{Abwandlung von Listing \ref{lst:rule_table}. Verwendet nun \texttt{tikz} um den Zellulären Automaten zu zeichnen.}
  \label{lst:rule_tikz}
\end{listing}
\todo{der code Listing \ref{lst:rule_tikz} ist cringe}

\begin{figure}
  \begin{minipage}{0.2\linewidth}
    \centering
    \large{Rule 30}
    \tikzCA{30}{50}{00000000000100000000000}{0.1}
    \bigskip{}
  \end{minipage}
  \begin{minipage}{0.2\linewidth}
    \centering
    \large{Rule 110}
    \tikzCA{110}{50}{00000000000100000000000}{0.1}
    \bigskip{}
  \end{minipage}
  \begin{minipage}{0.2\linewidth}
    \centering
    \large{Rule 69}
    \tikzCA{69}{50}{00000000000100000000000}{0.1}
    \bigskip{}
  \end{minipage}
  \begin{minipage}{0.2\linewidth}
    \centering
    \large{Rule 12}
    \tikzCA{123}{50}{00000000000100000000000}{0.1}
    \bigskip{}
  \end{minipage}
  \begin{minipage}{0.22\linewidth}
    \centering
    \large{Rule 95}
    \tikzCA{95}{50}{00000000000100000000000}{0.1}
  \end{minipage}
  \begin{minipage}{0.22\linewidth}
    \centering
    \large{Rule 28}
    \tikzCA{28}{50}{00000000000100000000000}{0.1}
  \end{minipage}
  \begin{minipage}{0.22\linewidth}
    \centering
    \large{Rule 90}
    \tikzCA{90}{50}{00000000000100000000000}{0.1}
  \end{minipage}
  \begin{minipage}{0.22\linewidth}
    \centering
    \large{Rule 102}
    \tikzCA{102}{50}{00000000000100000000000}{0.1}
  \end{minipage}
  \label{fig:ca_tikz}
  \caption{Eine Sammlung von Zellulären Automaten, welche mit \texttt{tikz} und \pythontex{} erzeugt wurde. Alle hatten den gleichen Startwert (einzelne lebende Zelle in der Mitte) und wurden für 50 Generationen simuliert. Der Code aus Listing \ref{lst:rule_tikz} wurde hierfür verwendet.}
\end{figure}
%
Es ist auch möglich komplexere Zelluläre Automaten zu simulieren und danach zu visualisieren.
%
Ein möglicher Ansatz ist in Listing \ref{lst:gameoflife} zu sehen.
%
Dort wird zuerst mit einer Funktion eine List der Generationen erzeugt.
%
Außerdem ist eine Funktion definiert, die ein \texttt{Board} Objekt nimmt und dieses mit \texttt{tikz} darstellt.
%
Wir können das dann mit \verb#\py{format_board_tikz(<gewollte generation>)}# wieder aufrufen.
%
So kann man zum Beispiel einen Zyklus eines Gliders in Abbildung \ref{fig:gameoflife_glider} sehen.
\begin{listing}
  \centering
  \begin{pyblock}
Board = List[List[bool]]

def gameoflife(seed: Board, num_gens: int) -> List[Board]:
    gens = [seed]
    height = len(seed)
    width = len(seed[0])
    for _ in range(num_gens - 1):
        new_gen = [list(r) for r in gens[-1]]
        for x, col in enumerate(gens[-1]):
            for y, cell in enumerate(col):
                alive = 0
                for xi in range(x - 1, x + 2):
                    for yi in range(y - 1, y + 2):
                        if gens[-1][xi % height][yi % width] and not (
                            xi == x and yi == y
                        ):
                            alive += 1
                if alive == 2 and cell:
                    new_gen[x][y] = True
                elif alive == 3:
                    new_gen[x][y] = True
                else:
                    new_gen[x][y] = False
        gens.append(new_gen)

    return gens


def format_board_tikz(board: Board, scale: float = 1) -> str:
    height = len(board)
    width = len(board[0])

    out = "\\begin{tikzpicture}\n"
    out += f"\\draw[step={scale}] (0,0)" \
           + f"grid ({width * scale},{height * scale});\n"
    for i, col in enumerate(board):
        for j, cell in enumerate(col):
            if cell:
                y = (height - i - 1) * scale
                x = j * scale
                out += f"\\fill[black] ({x}, {y})" \
                       + f" rectangle ({x + scale},{y + scale});\n"

    out += "\\end{tikzpicture}\n"
    return out
  \end{pyblock}
  \caption{Code um Conway's Game of Life zu simulieren.}
  \label{lst:gameoflife}
\end{listing}

\begin{pycode}
glider = [
    [0, 0, 0, 0, 0, 0, 0],
    [0, 0, 1, 0, 0, 0, 0],
    [0, 0, 0, 1, 0, 0, 0],
    [0, 1, 1, 1, 0, 0, 0],
    [0, 0, 0, 0, 0, 0, 0],
    [0, 0, 0, 0, 0, 0, 0],
    [0, 0, 0, 0, 0, 0, 0],
]

beacon = [
    [0, 0, 0, 0, 0, 0],
    [0, 1, 1, 0, 0, 0],
    [0, 1, 1, 0, 0, 0],
    [0, 0, 0, 1, 1, 0],
    [0, 0, 0, 1, 1, 0],
    [0, 0, 0, 0, 0, 0],
]

blinker = [
    [0, 0, 0, 0, 0],
    [0, 0, 1, 0, 0],
    [0, 0, 1, 0, 0],
    [0, 0, 1, 0, 0],
    [0, 0, 0, 0, 0],
]

glidergens = gameoflife(glider, 5)
beacongens = gameoflife(beacon, 5)
blinkergens = gameoflife(blinker, 5)
\end{pycode}

\begin{figure}
  \centering
  \begin{minipage}{0.19\linewidth}
    \centering
    \py{format_board_tikz(glidergens[0], 0.3)}
  \end{minipage}
  \begin{minipage}{0.19\linewidth}
    \centering
    \py{format_board_tikz(glidergens[1], 0.3)}
  \end{minipage}
  \begin{minipage}{0.19\linewidth}
    \centering
    \py{format_board_tikz(glidergens[2], 0.3)}
  \end{minipage}
  \begin{minipage}{0.19\linewidth}
    \centering
    \py{format_board_tikz(glidergens[3], 0.3)}
  \end{minipage}
  \begin{minipage}{0.19\linewidth}
    \centering
    \py{format_board_tikz(glidergens[4], 0.3)}
  \end{minipage}
  \caption{Die ersten fünf Generationen eines Gliders in Game of Life. Erzeugt mit dem Code aus Listing \ref{lst:gameoflife}.}
  \label{fig:gameoflife_glider}
\end{figure}

\section{Shits and giggles}
\begin{itemize}
  \item funny stuff
  \item pure latex CA
\end{itemize}

\bibliographystyle{plain}
\bibliography{\jobname{}}
\todo{referenzen}

\end{document}
