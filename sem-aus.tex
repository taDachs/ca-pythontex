\documentclass[11pt,a4paper]{article}

\usepackage[T1]{fontenc}
\usepackage[ngerman]{babel}
\usepackage[utf8]{inputenc}

\usepackage{pythontex}
\usepackage{xcolor}
\usepackage{url}

\usepackage{amsmath}
\usepackage{float}
% \usepackage{mathtools}

\usepackage[tt=false]{libertine}   % !!!!! das muss man nicht nutzen
\usepackage[libertine]{newtxmath}  % !!!!! das muss man nicht nutzen
%\usepackage[supstfm=libertinesups,supscaled=1.2,raised=-.13em]{superiors} % params taken from doc

%
%\usepackage{tgpagella}
%\usepackage[euler-digits]{eulervm}
%
\usepackage{csquotes}

\usepackage{microtype}

\usepackage{fancyvrb}

%\usepackage{graphicx}  % !!!!! benutzen, wenn Grafiken eingebunden werden sollen

\usepackage{booktabs}
\usepackage[shortlabels]{enumitem}
\setlist{noitemsep}

\usepackage{titlesec}
\usepackage{tcolorbox}
\tcbuselibrary{listingsutf8}

\usepackage{bbold}
\newcommand{\Z}{\mathbb{Z}}

\usepackage{hologo}

\definecolor{codegreen}{RGB}{0, 128, 0}
\definecolor{codegray}{rgb}{0.5,0.5,0.5}
\definecolor{codepurple}{rgb}{0.58,0,0.82}
\definecolor{codered}{RGB}{192, 53, 53}

\lstdefinestyle{mystyle}{
    commentstyle=\color{codegreen},
    keywordstyle=\color{codegreen},
    numberstyle=\tiny\color{codegray},
    stringstyle=\color{codered},
    basicstyle=\ttfamily,
    breakatwhitespace=false,
    breaklines=true,
}

\lstset{style=mystyle}

%-----------------------------------------------------------------------------
% für das Deckblatt

\usepackage{tikz}

\newcommand{\teilnehmername}{Maximilian Schik} % !!!!!
\newcommand{\teilnehmermatrnr}{2209036}        % !!!!!
\newcommand{\seminarart}{\LaTeX-Projektarbeit} % !!!!!
\newcommand{\seminarlp}{2 LP}                  % !!!!!
\newcommand{\seminarjahr}{2022}                % !!!!!
%-----------------------------------------------------------------------------
\newcommand{\meta}[1]{$\langle$\textit{#1}$\rangle$}
\newcommand{\paket}[1]{\texttt{#1}}
\newcommand{\prgname}[1]{\texttt{#1}}

%-----------------------------------------------------------------------------
\author{Maximilian Schik}
\title{PythonTex}

%-----------------------------------------------------------------------------
\pyc{from typing import List}
\newcommand{\tableCA}[3]{\py{rule(#1, #2, "#3")}}
\newcommand{\tikzCA}[4]{\py{rule_tikz(#1, #2, "#3", #4)}}
\newcommand{\toUpperCase}[1]{\py{"#1".upper()}}

\newcommand{\pythontex}{\textit{Python\TeX{}}}
\newcommand{\todo}[1]{\marginpar{\textcolor{red}{\textbf{TODO:} #1}}}

%=============================================================================
\begin{document}
%=======================================================================
% Anfang erste Seite
{\thispagestyle{empty}\large\sffamily\raggedright
%
\begin{tikzpicture}[remember picture,overlay]
  \coordinate[xshift=5mm,yshift=-5mm] (NW) at (current page.north west) {};
  \coordinate[xshift=-5mm,yshift=-5mm] (NE) at (current page.north east) {};
  \coordinate[xshift=-5mm,yshift=13mm] (SE) at (current page.south east) {};
  \coordinate[xshift=5mm,yshift=13mm] (SW) at (current page.south west) {};

  \draw[line width=0.25pt] (NW)
    [rounded corners=5mm] -- (NE)
    [sharp corners] -- (SE)
    [rounded corners=5mm] -- (SW)
    [sharp corners] -- cycle
  ;
\end{tikzpicture}
%
\unskip % keine Ahnung warum das nötig ist
\noindent \textbf{\Large \seminarart\ (\seminarlp)}
\\[\baselineskip]
%
% \\[1ex]
%
im Sommersemester \seminarjahr

\vspace*{3\baselineskip}

\noindent \textbf{\Large Pyt"|hon\TeX{}} \\[\baselineskip]
%
von \textbf{\teilnehmername}, Matr.nr.~\teilnehmermatrnr

\vspace*{3\baselineskip}

% \noindent \textbf{\Large PythonTex} \\[\baselineskip]
%
% nachfolgend ein Beispiel für Pro-/Seminare, für Konferenzbeiträge, Buchausschnitte, ...
% für LaTeX bitte was passendes ersetzen !!!!!
%
Thomas Worsch\\[1ex]
%
\LaTeX, \texttt{beamer}, \texttt{tikz} und Co.\\[1ex]
%
% Zentralblatt für historischen ÖPNV, Band \textbf{42}, S.~123-456
}
\clearpage
% Ende erste Seite
%=======================================================================
% Anfang zweite Seite
{\thispagestyle{empty}\raggedright

\noindent \textbf{\Large Erklärung}\\[1ex]
gemäß \S 6 (7) der Prüfungsordnung Informatik (Bachelor) 2015: % !!!!! oder (Master)    oder \S 6 (11) (Bachelor) 2008
\\[\baselineskip]

\noindent
Ich versichere wahrheitsgemäß, die vorliegende Arbeit selbstständig
angefertigt, alle benutzten Hilfsmittel vollständig und genau
angegeben und alles kenntlich gemacht zu haben, was aus Arbeiten
anderer unverändert oder mit Abänderungen entnommen wurde.

\vspace*{30mm}
\noindent
\begin{tabular}{@{}l}
  \hline
   \\[-1ex]
  \hbox to 0.6\textwidth{(\teilnehmername, Matr.nr.~\teilnehmermatrnr) \hss}
\end{tabular}
}
\clearpage
% Ende zweite Seite
%=======================================================================

%-----------------------------------------------------------------------------
\section{Einleitung}
\pythontex{} \cite{pythontex-paper} ist ein \LaTeX{} Paket das es einem erlaubt Python Code innerhalb eines \LaTeX{} Dokumentes zu verwenden und dessen Ausgabe auch in dem Dokument darzustellen.
%
Das Paket erlaubt es dem Nutzer den Code direkt neben seiner Ausgabe zu speichern.
%
Dadurch kann man Dokumente verbreiten die direkt den Code beinhalten, mit dem die Grafiken oder sonstige Daten im Dokument selbst erzeugt wurden.
%
Das erzeugt einen hohen Grad an Reproduzierbarkeit.
%
Normalerweise wird Reproduzierbarkeit nur erreicht, indem man den verwendeten Programm Code seperat verbreitet.
%
Dies ist aber nicht immer der Fall (was Projekte wie \emph{Papers without Code}\cite{paperswithoutcode} oder Studien wie \cite{reproducible-study} und \cite{StateofReproducible} nur unterstreichen), vorallem in den Bereichen Künstlicher Intelligenz und Data Science, welche fast vorwiegend Python verwenden.
\\
%
Die Einbindung des Codes in das Dokument ist vielfältig, so kann man zum Beispiel mit Hilfe des \verb+\py+ Befehls direkt Python Code im Dokument ausführen.
%
\verb#\py{1 + 2 + 3}# gibt einem als Ausgabe \texttt{\py{1 + 2 + 3}}. Es erlaubt auch direkt \LaTeX{} Befehle mit auszugeben.
%
\verb#\py{r"\textit{Hello World}"}# ergibt \py{r"\textit{Hello World}"} (beachte die \textit{Italics}).
%
\\
%
\pythontex{} wurde hauptsächlich von Geoffrey M Poore entwickelt und maintained.
%
Es ist verfügbar in \texttt{\TeX{} Live} und \texttt{MiK\TeX{}}.
%
\section{Das Paket \pythontex{}}
\pythontex{} stellt eine Reihe an nützlichen Umgebungen und Befehlen bereit, hier werden allerdings nur die wichtigsten behandelt.
%
Alle Befehle und Umgebungen werden für den interessierten Leser in \cite{pythontex-paper} sehr ausführlich beschrieben.
%
Die wichtigsten sind:
%
\begin{itemize}
  \item \verb#\py# - Führt den gegebenen Code aus und gibt die Ausgabe zurück.
  \item \verb#\pyc# - Führt den gegebenen Code aus, gibt nur \pyb{print} Aufrufe aus. Im Gegensatz zu \verb#\py# wird nicht die direkte Ausgabe mit zurückgegeben.
  \begin{itemize}
    \item \verb#\py{"foo".upper()}# \textrightarrow\ \py{"foo".upper()}
    \item \verb#\pyc{"foo".upper()}# \textrightarrow\  Keine Ausgabe, um die gleiche Ausgabe zu erhalten müsste \verb#\pyc{print("foo".upper())}# aufgerufen werden.
  \end{itemize}
  \item \verb#\pyb# - Führt den gegebenen Code aus und formatiert ihn. Die Ausgabe kann später mittels \verb#\printpythontex# aufgerufen werden.
\end{itemize}
%
Es gibt noch einige Umgebungen, diese entsprechen teilweise einem der vorangegangen Befehle.
\begin{itemize}
  \item \texttt{pycode} - führt den gegebenen Code aus. Gibt nur \pyb{print} Aufrufe aus.
  \item \texttt{pyblock} - führt den gegebenen Code aus und stellt den Code als Listing mittels \texttt{pygments} dar. Ähnlich zu \verb#\pyb# kann mit \verb#\printpythontex# die Ausgabe ausgegeben werden.
  \item \texttt{pyconsole} - führt den gegebenen Code in einer interaktiven Konsolen Sitzung aus.
\end{itemize}
%
Man kann auch Python Aufrufe in einen \LaTeX{} Befehl kapseln.
%
So würde zum Beispiel \verb~\newcommand{\toUpperCase}[1]{\py{"#1".upper()}}~ einen Befehl erzeugen, der eine beliebige Zeichenkette nimmt und diese in eine Zeichenkette aus Großbuchstaben umformt.
%
Diesen Befehl kann man dann einfach mit \verb#\toUpperCase{foo}# aufrufen um die Ausgabe ``\toUpperCase{foo}`` zu erhalten.
%
Dabei sollte man stehts beachten das sich sämtlicher Python Code einen Namespace teilt, daher sollte man nach Möglichkeit Logik immer in Funktionen kapseln.
%
\\
%
Das letztendliche Erzeugen des Dokuments mit \pythontex{} besteht aus wenigen Schritten.
%
Man verwendet das \pythontex{} Paket mittels \verb#\usepackage{pythontex}# in einem Dokument und ruft Python mit einem der beschriebenen Befehle innerhalb des Dokuments auf.
%
Dieses kompiliert man dann mit einer der unterstützten \LaTeX{} Engines (pdf\TeX{}, Xe\TeX{} oder Lua\TeX{}).
%
Danach ruft man \texttt{pythontex} auf das Dokument auf.
%
Das nimmt dann den Python Code aus dem Dokument und führt ihn aus.
%
Die Ergebnisse werden in einem Unterverzeichnis gespeichert.
%
Mit einem erneuten Aufruf der \LaTeX{} Engine werden nun diese Ergebnisse in das Dokument eingebunden und die finale PDF wird erzeugt.
%
Für dieses Dokument wurde das alles in einer Makefile zusammengefasst, welche in Listing \ref{lst:makefile} zu sehen ist.
%
Da eine Abgabe eines Dokuments mit \pythontex{} eventuell nicht überall möglich ist (sei es aus kompatibilitäts oder Sicherheits Gründen) bietet \pythontex{} auch eine Möglichkeit ein Dokument zu ``depythonfizieren``, und zwar mittels des \texttt{depythontex} Programs.
%
Dieses ist allerdings nicht mit Benutzer-definierten \LaTeX{} Befehlen, welche \pythontex{} verwenden, kompatibel, daher ist das für dieses Dokument nicht möglich.
\\
%
Ein Nachteil von \pythontex{} ist allerdings das wegfallen von Linting- und Typechecking Tools für Python.
%
Da so etwas wie \texttt{mypy} nicht in der Lage ist zwischen Python und \LaTeX{} Code zu unterscheiden kann es schnell schwer werden größere Projekte im Dokument selbst zu entwickeln.
%
Als Lösung dafür bietet \pythontex{} die Möglichkeit Code aus externen Dateien zu importieren, mit denen \texttt{mypy} wiederum arbeiten kann.
%
Ein weiterer Nachteil ist von \pythontex{} ist ein Python Problem.
%
Das folgende Code Stück würde zu einem Fehler führen wenn man es als normales Python Programm ausführt.
%
\begin{lstlisting}[language=Python]
foo()

def foo():
  print("bar")
\end{lstlisting}
%
Das Problem hier ist das Python kein Hoisting hat (Deklarationen werden an die Spitze ihres Scopes gezogen).
%
Daher ist es manchmal etwas schwer die Ausgabe eines Listings vor dem Listing selbst zu haben (als Beispiel mal Abbildung \ref{fig:rule_30_table} und Listing \ref{lst:rule_table}).
%
Man kann zwar eine Kopie des Codes in das Präamble schreiben, allerdings endet das in dupliziertem Code, genau das was \pythontex{} verhindern will.
%
\begin{listing}[t]
  \begin{lstlisting}[language=make,columns=flexible]
build:
	pdflatex -interaction=nonstopmode sem-aus.tex
	pythontex sem-aus.tex
	pdflatex -interaction=nonstopmode sem-aus.tex
  \end{lstlisting}
  \caption{Makefile für dieses Dokument}
  \label{lst:makefile}
\end{listing}
%
\\
%
\pythontex{} unterstützt neben Python auch noch Julia und Ruby, allerdings wurden diese Sprachen im Rahmen dieser Ausarbeitung nicht behandelt.
\section{Ein paar Beispiele}
\subsection{Zelluläre Automaten als Tabelle}
Ein weiterer Vorteil von \pythontex{} ist die Möglichkeit, das Ergebnis von Algorithmen/Simulationen direkt zu visualisieren.
%
Man kann zum Beispiel die ersten Generationen eines elementaren Zellulären Automatens in einer Tabelle darstellen.
%
Dies ist möglich da \pythontex{} die Ausgabe direkt als \LaTeX{} interpretiert und demnach auf formatiert.
%
So kann sich innerhalb einer \texttt{pyblock} Umgebung eine Funktion definieren die einen Zellulären Automaten simuliert und die Genrationen danach in eine Tabelle packt.
%
Ein mögliches Beispiel dafür wäre Listing \ref{lst:rule_table}, das Ergebnis ist in Abbildung \ref{fig:rule_30_table} zu sehen.
%
Für alle Python Listings wurde die \texttt{pyblock} Umgebung verwendet.
%
\begin{listing}[h]
  \centering
  \begin{pyblock}
def rule(n, t, seed):
    seed = "0" * t + seed + "0" * t
    rule = f"{n:b}".zfill(8)[::-1]
    gens = [seed]
    out = f"\\begin{{tabular}}{{ {'c ' * len(seed)} }}\n"
    for i in range(t):
        prev_gen = gens[i]
        new_gen = []
        for j in range(1, len(prev_gen) - 1):
            new_gen.append(rule[int(prev_gen[j-1:j+2], 2)])
        gens.append("".join(new_gen))
    for i, gen in enumerate(gens):
        out += " &  " * i
        out += " & ".join(gen)
        out += " &  " * i + r"\\" + "\n"
    out += r"\end{tabular}"
    return out
  \end{pyblock}
  \caption{Python Code zum Erzeugen einer Tabelle, die die ersten \texttt{t} Generationen des eines Zellulären Automatik für eine gegebene Regel \texttt{n} mit dem Startwert \texttt{seed} enthält.}
  \label{lst:rule_table}
\end{listing}
%
\begin{figure}[H]
  \centering
  \tableCA{30}{3}{00101101001}
  \caption{Ausgabe von Listing \ref{lst:rule_table} mit \texttt{n = 30}, \texttt{t = 3} und \texttt{seed = "00101101001"}. Der \LaTeX{} Befehl war \texttt{\textbackslash tableCA\{30\}\{3\}\{00101101001\}}.}
  \label{fig:rule_30_table}
\end{figure}

\subsection{Fortgeschrittene Beispiele}
Alternativ kann man auch anstatt einer Tabelle ein \texttt{tikz} Bild zeichnen.
%
Eine Beispiel dafür gibt es in Listing \ref{lst:rule_tikz} zu sehen.
%
Das \texttt{tikz} Bild kann man dann mit dem \LaTeX{} Befehl \texttt{\textbackslash tikzCA\{<Regel>\}\{<Num. Gens>\}\{<seed>\}\{<Skalierung>\}} aufrufen, welcher ähnlich zu \verb#\tableCA# gebunden wurde.
%
\begin{listing}
  \centering
  \begin{pyblock}
def rule_tikz(n, t, seed, scale):
  rule = f"{n:b}".zfill(8)[::-1]
  gens = [seed]
  for i in range(t):
      prev_gen = "0" + gens[i] + "0"
      new_gen = []
      for j in range(1, len(prev_gen) - 1):
          new_gen.append(rule[int(prev_gen[j-1:j+2], 2)])
      gens.append("".join(new_gen))
  out = "\\begin{tikzpicture}\n"
  out += f"\\draw[step={scale}] (0,0)" \
         + f"grid ({len(seed) * scale},{(t + 1) * scale});\n"
  for i, gen in enumerate(gens):
      for j, cell in enumerate(gen):
          if int(cell):
              y = (t - i) * scale
              x = j * scale
              out += f"\\fill[black] ({x}, {y})" \
                     + f" rectangle ({x + scale},{y + scale});\n"
  out += "\\end{tikzpicture}\n"
  return out
  \end{pyblock}
  \caption{Abwandlung von Listing \ref{lst:rule_table}. Verwendet nun \texttt{tikz} um den Zellulären Automaten zu zeichnen.}
  \label{lst:rule_tikz}
\end{listing}
%
\begin{figure}[H]
  \begin{minipage}{0.245\linewidth}
    \centering
    \large{Rule 12}
    \tikzCA{123}{50}{00000000000100000000000}{0.1}
  \end{minipage}
  \begin{minipage}{0.245\linewidth}
    \centering
    \large{Rule 30}
    \tikzCA{30}{50}{00000000000100000000000}{0.1}
  \end{minipage}
  \begin{minipage}{0.245\linewidth}
    \centering
    \large{Rule 69}
    \tikzCA{69}{50}{00000000000100000000000}{0.1}
  \end{minipage}
  \begin{minipage}{0.245\linewidth}
    \centering
    \large{Rule 110}
    \tikzCA{110}{50}{00000000000100000000000}{0.1}
  \end{minipage}
  \label{fig:ca_tikz}
  \caption{Eine Sammlung von Zellulären Automaten, welche mit \texttt{tikz} und \pythontex{} erzeugt wurde. Alle hatten den gleichen Startwert (einzelne lebende Zelle in der Mitte) und wurden für 50 Generationen simuliert. Der Code aus Listing \ref{lst:rule_tikz} wurde hierfür verwendet.}
\end{figure}
%
Es ist auch möglich komplexere Zelluläre Automaten zu simulieren und danach zu visualisieren.
%
Ein möglicher Ansatz ist in Listing \ref{lst:gameoflife} zu sehen.
%
Dort wird zuerst mit einer Funktion eine List der Generationen erzeugt.
%
Außerdem ist eine Funktion namens \texttt{format\_board\_tikz} definiert, die ein \texttt{Board} Objekt (ein 2D Array von Booleans oder Integern) nimmt und dieses mit \texttt{tikz} darstellt.
%
Wir können das dann mit \verb#\py{format_board_tikz(<gewollte generation>)}# wieder aufrufen.
%
So kann man zum Beispiel einen Zyklus eines Gliders in Abbildung \ref{fig:gameoflife_glider} erzeugen.
%
\begin{listing}[H]
  \centering
  \begin{pyblock}
def gameoflife(seed, num_gens):
    gens = [seed]
    height = len(seed)
    width = len(seed[0])
    for _ in range(num_gens - 1):
        new_gen = [list(r) for r in gens[-1]]
        for x, col in enumerate(gens[-1]):
            for y, cell in enumerate(col):
                alive = 0
                for xi in range(x - 1, x + 2):
                    for yi in range(y - 1, y + 2):
                        if gens[-1][xi % height][yi % width] and \
                           not (xi == x and yi == y):
                            alive += 1
                new_gen[x][y] = (alive == 2 and cell) or \
                                (alive == 3)
        gens.append(new_gen)
    return gens

def format_board_tikz(board, scale):
    height = len(board)
    width = len(board[0])
    out = "\\begin{tikzpicture}\n"
    out += f"\\draw[step={scale}] (0,0)" \
           + f"grid ({width * scale},{height * scale});\n"
    for i, col in enumerate(board):
        for j, cell in enumerate(col):
            if cell:
                y = (height - i - 1) * scale
                x = j * scale
                out += f"\\fill[black] ({x}, {y})" \
                       + f" rectangle ({x + scale},{y + scale});\n"
    out += "\\end{tikzpicture}\n"
    return out
  \end{pyblock}
  \caption{Code um Conway's Game of Life zu simulieren und mit \texttt{tikz} zu zeichnen.}
  \label{lst:gameoflife}
\end{listing}
%
\begin{pycode}
glider = [
    [0, 0, 0, 0, 0, 0, 0],
    [0, 0, 1, 0, 0, 0, 0],
    [0, 0, 0, 1, 0, 0, 0],
    [0, 1, 1, 1, 0, 0, 0],
    [0, 0, 0, 0, 0, 0, 0],
    [0, 0, 0, 0, 0, 0, 0],
    [0, 0, 0, 0, 0, 0, 0],
]

glidergens = gameoflife(glider, 5)
\end{pycode}

\begin{figure}[H]
  \centering
  \begin{minipage}{0.19\linewidth}
    \centering
    \py{format_board_tikz(glidergens[0], 0.3)}
  \end{minipage}
  \begin{minipage}{0.19\linewidth}
    \centering
    \py{format_board_tikz(glidergens[1], 0.3)}
  \end{minipage}
  \begin{minipage}{0.19\linewidth}
    \centering
    \py{format_board_tikz(glidergens[2], 0.3)}
  \end{minipage}
  \begin{minipage}{0.19\linewidth}
    \centering
    \py{format_board_tikz(glidergens[3], 0.3)}
  \end{minipage}
  \begin{minipage}{0.19\linewidth}
    \centering
    \py{format_board_tikz(glidergens[4], 0.3)}
  \end{minipage}
  \caption{Die ersten fünf Generationen eines Gliders in Game of Life. Erzeugt mit dem Code aus Listing \ref{lst:gameoflife}.}
  \label{fig:gameoflife_glider}
\end{figure}
%
Es ist auch möglich mit Bibliotheken wie \texttt{matplotlib} Graphen zu erzeugen und direk im Dokument darzustellen, allerdings geht dies über den direkten Weg, sondern es muss zuerst der Graph in einer seperaten Datei gespeichert werden, um ihn dann mit \verb#\includegraphics# im Dokument einzubinden.
%
Das wird allerdings nicht in dieser Ausarbeitung gezeigt, da ansonsten eine Abhängigkeit auf diese Bibliothek bestehen würde und hier nur die grundlegenen Features gezeigt werden sollen.

\section{Zusammenfassung}
\pythontex{} bietet dem Nutzer eine sehr einfach zu verwendende Schnittstelle an um Python mit \LaTeX{} zu mischen.
%
Dies kann zum Beispiel verwendet werden um spezielle Grafiken zu erzeugen oder Ergebnisse von komplexen Berechnungen oder Algorithmen direkt von im Dokument gezeigten Code abhängig zu machen.
%
Da die Ausgabe des Python Codes einfach nur ein String ist kann man quasi alles Erzeugen was in \LaTeX{} möglich ist.
%
Allerdings kann das mischen von Python Code und \LaTeX{} auch als Sicherheitslücke gesehen werden, was vor allem durch das nicht voll funktionsfähige \texttt{depythontex} Programm nicht immer behoben werden kann.


\clearpage

\bibliographystyle{plain}
\bibliography{sem-aus}

\end{document}
